\section{Further Studies}

From this basic version of SOM,
there have been some developments of extensions of this method.
One example is the time-adaptive self-organizing map (TASOM)~\cite{shah:2003},
which modifies SOM to have adaptive learning rates and neighborhood sizes.
Rather than having a global learning rate and neighborhood size,
every node has its own.
Some simulations show satisfactory results in the applications the authors considered.

Another extension is the growing self-organizing map (GSOM)~\cite{hsu:2003},
which modifies SOM to adaptively discover the optimal grid size.
While the algorithm is significantly more complicated and computationally heavy,
it often gives a better representation of the data geometry than the SOM\@.
GSOM was successfuly applied to leukemia microarray data
containing three types of leukemia,
and was able to determine three major and one minor clusters~\cite{hsu:2003}.

