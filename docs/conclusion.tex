\section{Conclusion}

SOM is a powerful and simple unsupervised learning algorithm
to visualize patterns in a dataset.
As discussed in Section~\ref{ssec:cl} and~\ref{ssec:som},
SOM can be thought of as an extension of an earlier work called vector quantization,
which tries to learn representative values of the data distribution,
with the key insight that it also takes advantage of the spatial topology of the nodes.
Section~\ref{ssec:colors} gives a toy example of organizing randomly generated colors
and we saw how the nodes became organized to detect certain types of colors.
Section~\ref{ssec:transfer} gives an application with a real dataset 
of organizing football players into tiers and we saw that a naive clustering algorithm on the raw data
was not sufficient for a meaningful result and that a more sophisticated method like SOM was needed.
Furthermore, SOM was able to truly learn based on the meaningful features and give reasonable ordering of the players.
